\section{Continuous distributions}

\begin{slide}
\begin{description}
\item[Probability Density Function]
A random vriable $X$ has PDF $f(x)$ if $$P(a\leq X \leq b) = \int_a^bf(x)\,dx$$ Also, $f(x)\geq 0$ and $\int_{\infty}^\infty f(x)dx = 1$ 
\item[Cumulative Density Function]
$F(x) = P(X\leq x) = \int\limits_{-\infty}^xf(t)dt$
$$f(x) = \dot{F}(x)$$
\item[Expectation] $E(X) = \int_{-\infty}^\infty xf_X(x)dx$
\item[Variance] $Var(X) = E[X - EX]^2 = EX^2 - (EX)^2$
\item[LOTUS]\footnote[3]{Law Of The Unconcious Statistician. True even for discrete case} $E(g(X)) = \int_{-\infty}^\infty g(x)f_X(x)dx$
\end{description}

\footnotetext{$P(X=constant)=0$ bacause its continuous and $\int_a^af(x)dx=0$}
\footnotetext{$EX^2 \equiv E(X^2)$}
\end{slide}

\begin{slide}[Uniform distribution]
\begin{shaded}
	\begin{center}
		A random point in an interval
	\end{center}
    $$X \in Unif(a,b)$$
    $$
    f_X(x) = {1\over b-a}  
    \quad 
    F(x)   = {x-a \over b-a}
    \quad
    E(X)   = {a+b \over 2} 
    \quad
    Var(X) = {a+b \over 2} 
    $$
\end{shaded}
\end{slide}

\begin{slide} [Normal Distribution]
  \begin{shaded}
    %{\noindent{}definition}
    $$X \in N(\mu,\sigma)$$
    $$
    f_X(x) = {1\over \sigma \sqrt{2\pi}} e^{-{(x-\mu)^2 \over 2\sigma^2}} 
    \quad
    E(X)   = \mu
    \quad
    Var(X) =\sigma^2
    \quad 
    F(X)   = {1\over 2}\left[1+erf\left({x-\mu \over \sigma\sqrt{2}}\right)\right]
    $$
  \end{shaded}

$Var(X+c) = Var(X)$, $Var(cX) = c^2Var(X)$, $Var(X+Y) \neq Var(X) + Var(Y)$

\end{slide}

\begin{slide}[Exponential Distribution]
 \begin{shaded}
    %{\noindent{}definition}
    $$X \in Expo(\lambda)$$
    $$
    f_X(x) = \lambda e^{-\lambda x} 
    \quad
    E(X)   = {1 \over \lambda}
    \quad
    Var(X) = {1 \over \lambda^2}
    \quad 
    F(X)   = 1-e^{\lambda x}
    $$
  \end{shaded}
Memoryless property: $P(X\geq s+t|X\geq s) = P(X\geq t)$. amount of time waited doesnt matter. Only exponential is continuous distribution having this property.
  $Y = \lambda X \implies Y\in Expo(1)$

\end{slide}


\begin{slide} [Beta Distribution]
 \begin{shaded}
    %{\noindent{}definition}
    $$X \in Beta(a,b)$$
    $$
    f_X(x) = {1\over B(a,b) } x^{a-1}(1-x)^{b-1} 
    \quad
    E(X)   = {a\over a+b}
    \quad
    Var(X) = {ab\over (a+b)^2(a+b+1)}
    $$
    $$ 
    B(a,b)   = {\Gamma(a)\Gamma(b)\over\Gamma(a+b)}
    $$
  \end{shaded}
Flexible distributions in interval $(0,1)$. Often used as prior for a parameter. 
Beta is conjugte prior of BErnoulli distribution.
\end{slide}

\begin{slide}[Gamma Distribution]
 \begin{shaded}
    %{\noindent{}definition}
    $$X \in Gamma(a,\lambda)$$
    $$
    f_X(x) = {1 \over \Gamma(\alpha)} \lambda^a x^{a-1}e^{-\lambda x}  
    \quad
    E(X)   = {a\over \lambda}
    \quad
    Var(X) = {a\over \lambda^2}
    %\quad 
    %F(X)   = {}
    $$
  \end{shaded}
Gamma Function
  \begin{align*}
\Gamma(a) &= \int_0^\infty x^a e^{-x}{dx\over x}. \text{ for real } a>0
\\
\Gamma(n) &=(n-1)!
\\
\Gamma(x+1) &= x\Gamma(x)
\\
\Gamma\left({1/2}\right) &=  \sqrt{\pi}
\end{align*}
\end{slide}

\begin{slide}[Change of variables]
\noindent{}If $y = h(x)$, then, 
\begin{shaded}
  {\huge
  $$f_X(x) = f_Y(y){dy \over dx}$$
  }
\end{shaded}
\noindent In multidimensional case, ${dy\over dx}$ is jacobian matrix
\end{slide}

\begin{slide}%maximus}
  %\begin{multicols*}{3}
Beta is conjugate prior to Binomial. If $X|p \in Bin(n,p)$, and lets assume $p \in Beta(a,b)$ (prior), what is $p|X$ (posterior)? 
\begin{align*}
f&(p| X=k) 
\\
&= {P(X=k|p)f(p) \over P(X=k)} 
\\
&= { n\choose k} p^k(1-p)^{n-k} {p^{a-1}(1-p)^{b-1} \over B(a,b) P(X=k)} 
\\
&\propto p^{a+k-1}(1-p)^{b+n-k-1}
\end{align*}
$p|X \in Beta (a+X,b+n-X)$
%\vfill
  %\end{multicols*}
\end{slide}%maximus}
